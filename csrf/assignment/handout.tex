% !TEX encoding = UTF-8 Unicode
\documentclass{./handout}

\usepackage{hyperref}
\usepackage{verbatim}

\title{CSRF}
\assignmentno{1}

\date{Tuesday, 4 Jun. 2014}

\begin{document}
\maketitle


\section*{Structure}
The following assignments have been written short on hints. This is due 
to the fact that I don't know on which level your skills are.
If you are stuck on something, please don't hesitate to ask me. It would
be a pity, if you would spend half the time on figuring out JavaScript
syntax subtleties.
It may be helpful for the assignments to revisit the presentation slides.
I have made them available at
\href{http://cmichi.github.io/talks/csrf/talk/}{http://cmichi.github.io/talks/csrf/talk/}.

To give a rough structure of the practical assignments:

\begin{tabular}{{l}{l}{l}{l}{l}}
Part 1 & ca. 30 min	\\
Presentation Solution & ca. 10 min\\
\\
Part 2 & ca. 60 min	\\
Presentation Solution  & ca. 20 min	\\
\end{tabular}


\task{Metasploitable}{30}
These tasks will be executed using Metasploitable. Please start the
application and make sure you have access to the webserver of the virtual
machine.

\begin{subtask}
\paragraph{Damn Vulnerable Web App}
Direct your browser to the IP-address of the Metasploitable image,
e.g. \href{http://192.168.1.103/dvwa}{http://192.168.1.103/dvwa}. 
Configure Security Level \texttt{low} as described in the preparation 
document.

Try to exploit the CSRF vulnerability available via the navigation bar
on ``\textt{/dvwa/vulnerabilities/csrf/}''.
The button ``\texttt{View Source}'' might provide additional information.
Your exploit should be a HTML page containing only HTML markup code. Once 
this markup is interpreted by a browser, a HTTP request should be send by 
the browser in order to set a new password.
Think about which typical HTML elements execute HTTP requests in order to
load a further resource.
\end{subtask}

\begin{subtask}
\paragraph{Damn Vulnerable Web App}
Configure Security Level \texttt{medium} and try to exploit the hardened 
CSRF vulnerability.
Your exploit should consist of HTML markup and JavaScript code this time.
\end{subtask}

\begin{subtask}
\paragraph{TWiki}
Write an exploit which creates a new wiki page. To achieve this, prepare 
a HTML page with the embedded exploit. The exploit should not rely on 
a user being logged in. Feel free to use HTML and/or JavaScript.
%subtask mit xss?
\end{subtask}

\newpage
\task{Advanced CSRF}{60}
These tasks will be done using the advanced-csrf web application.
Please see the 
\href{https://moodle.uni-ulm.de/course/view.php?id=1312}{Moodle platform} for 
the advanced-csrf.zip. Please unzip the archive to your web server executable
directory.

You will then need to create a database ``advanced-csrf'' within MySQL and 
import the dump.sql into your MySQL database.
You might then need to adapt the MySQL credentials within the
``advanced-csrf/mysql.inc.php''.

\begin{subtask}
The site ``/ex2-1/leaking.php'' provides a way of leaking textual information.
Suppose somebody wants to frame a certain person of leaking information.
They craft a mail and use social engineering mechanisms as a mean to
get a victim to visit a website which contains a malicious CSRF attack.

How would the exploit on such a website need to be written as a mean
to submit the form with textual information, without the user noticing?
Use JavaScript and HTML to assemble such a website.
\end{subtask}

\begin{subtask}
%\paragraph{Token Synchronizer Pattern}
The site ``/ex2-2/index.php'' contains a form which the developers tried to
make safe against CSRF attacks using the Token Synchronizer Pattern.
%Since the server does not hold any state, the pattern has been implemented
%in a way that state is only saved on the client using cookies. 
%The pattern, however, has been implemented in a weak way.
%Whilst implementing the pattern a fatal error has been made.

Try to exploit the application in a way that when a visitor visits 
``/ex2-2/index.php'' he automatically votes for banana without noticing it.
\end{subtask}

\begin{comment}
\begin{subtask}
\paragraph{Token Synchronizer Pattern}
The implementation of the Token Synchronizer Pattern has been improved.
Find a way to still exploit the web page.

``Usually, the tokens should be some function (with a secret key - known only
to the server; e.g., MAC) of the cookie! not the cookie.

Than the flow is as follows: 1. Client sends the server request with a
cookie. 2. Server returns a web page with CSRF token(s) for different
purposes (e.g., forms or just a simple get requests via the URL). 3. The
client performs some action (via POST or GET) and sends request with the
token (in the request body or in the URL) and with the cookie. 4. The
server is stateless, but it can verify that the request was sent by the
same client by calculating the function (with the secret key that the
server knows) on the cookie (or on part of it), and comparing the output
with the token.

In the case of CSRF, the cookie is automatically appended to the request by
the browser, but the attacker (that probably even doesn't know the cookie)
cannot add the corresponding tokens.''
\end{comment}

% a fixed token is generated.
% xss on other page
% create iframe to vote page and read the token
%\begin{subtask}
%\end{subtask}

%\begin{subtask}
%referrer check
%The site from the previous task has been improved and now also checks the 
%\end{subtask}


\begin{subtask}
%multi step form with json of all answers on last step
%\paragraph{Multi-Step Form}
The site ``/ex2-3/index.html'' contains a multi-step form for a subscription 
service. Your goal is to prepare an exploit which will lead the browser who 
interprets the code into submitting the completed survey with options
of your choice.
\end{subtask}

\begin{subtask}
The voting page has an administration interface. Assume the administrators
are logged into it and have full access to the JSON API of the
administration interface.
For example, this command will reset the voting:

\begin{lstlisting}
POST /ex2-3/api/results HTTP/1.1
Host: localhost

{
  "action": "reset-voting",
  "value": "all"
}
\end{lstlisting}

The administrators are very careful and have turned off JavaScript within
their browsers.
However, they still get send a faked mail with a link to a pretty website 
-- which they of course visit.

How can you write a -- purely HTML based -- exploit, which will send 
manipulated calls to the API?
Please write an exploit which makes a POST request with a JSON payload.
%and find a way to inject this exploit into the voter /home/ page.
\end{subtask}

%Some RESTful Webservices provide the possibility
%step1: delete user account, input mailaddress
%step2: input that you read the text
%step3: are you sure?
%multi step form, each form needs to be submitted and directs
%to another web page, state is saved server-side
%last site contains only a button "i'm sure" and "cancel"
%"i'm sure" makes an ajax call to a restful webservice /api/members/123/
%csrf against restful webservice
%* restricted to HTTP method PUT
%* restricted to AJAX (X-Requested-With: XMLHttpRequest)
%* media-tzype of forms is URL encoded || multi-art || text/plain, restrict to application/json

\begin{comment}
\begin{subtask}
%\paragraph{Semi-Blind, Multi-Step Form}
The site from the previous task has been changed, it is now a multi-step
survey -- which is still exploitable.
Develop an exploit which subsequently fills each step of the survey with 
data and eventually submits the completed survey.
%multi-step form with state on each step
%semi-blind
\end{subtask}
\end{comment}

%hard task: referrer spoofing
% hard task: session token saved in cookie, but appended as header field

\center
\ \\
\texttt{\$ ./happy\_hacking -\--now -\--mood ":)"  }




\end{document}
