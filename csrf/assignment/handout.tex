% !TEX encoding = UTF-8 Unicode
\documentclass{./handout}

\usepackage{hyperref}

\title{CSRF}
\assignmentno{6}

\date{Tuesday, 4 Jun. 2014} % Abgabedatum

\begin{document}
\maketitle

\task{Metasploitable}{30}

\begin{subtask}
\paragraph{Damn Vulnerable Web App}
Direct your browser to the IP-address of the Metasploitable image,
e.g. \href{http://192.168.1.103/dvwa}{http://192.168.1.103/dvwa}. 
Configure Security Level \texttt{low} as described in the preparation 
document.

Try to exploit the CSRF vulnerability available via the navigation bar
on \href{http://192.168.1.103/dvwa/vulnerabilities/csrf/}
	{http://192.168.1.103/dvwa/vulnerabilities/csrf/}.
The button \texttt{View Source} might provide additional information.
Your exploit should consist solely of HTML markup code. Once this markup
is interpreted by a browser a HTTP request should be send by the browser
in order to set a new password.
\end{subtask}

\begin{subtask}
\paragraph{Damn Vulnerable Web App}
Configure Security Level \texttt{medium} and try to exploit the hardened 
CSRF vulnerability.
Your exploit may consist of HTML markup and JavaScript code.
\end{subtask}

\begin{subtask}
\paragraph{TWiki}
Write an exploit which creates a new wiki page. To achieve this, prepare 
a HTML page with the embedded exploit. The page should execute the exploit
once it is opened within a browser.
\end{subtask}


\task{TWiki}{30}
\begin{subtask}
in conjuction with xss. scenario: token is saved in cookie and we use xss
to read it, create the form, fill it in.
\end{subtask}

\begin{subtask}
multi step form with json on last step
\end{subtask}

\begin{subtask}

multi-step form with state on each step

semi-blind
\end{subtask}

\end{document}
