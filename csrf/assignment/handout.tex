% !TEX encoding = UTF-8 Unicode
\documentclass{./handout}

\usepackage{hyperref}
\usepackage{verbatim}

\title{CSRF}
\assignmentno{1}

\date{Tuesday, 4 Jun. 2014}

\begin{document}
\maketitle


\section*{Structure}
The following assignments have been written short on hints. This is due 
to the fact that I don't know on which level your skills are.
If you are stuck on something please don't hesitate to ask me. It would
be a pity if you would spend half the time on figuring out JavaScript
syntax subtleties.

To give a rough structure of the practical assignments:

\begin{tabular}{{l}{l}}
10.00-10.30	& Assignment 1\\
10.30-10.40	& Presentation Solution\\
\ \\
10.40-11.40	& Assignment 2\\
11.40-12.00	& Presentation Solution\\
\end{tabular}



\task{Metasploitable}{30}
These tasks will be executed using Metasploitable. Please start the
application and make sure you have access to the webserver.

\begin{subtask}
\paragraph{Damn Vulnerable Web App}
Direct your browser to the IP-address of the Metasploitable image,
e.g. \href{http://192.168.1.103/dvwa}{http://192.168.1.103/dvwa}. 
Configure Security Level \texttt{low} as described in the preparation 
document.

Try to exploit the CSRF vulnerability available via the navigation bar
on \href{http://192.168.1.103/dvwa/vulnerabilities/csrf/}
	{http://192.168.1.103/dvwa/vulnerabilities/csrf/}.
The button \texttt{View Source} might provide additional information.
Your exploit should be a HTML page containing only HTML markup code. Once 
this markup is interpreted by a browser a HTTP request should be send by 
the browser in order to set a new password.
Think about which typical HTML elements execute HTTP requests in order to
load a further resource.
\end{subtask}

\begin{subtask}
\paragraph{Damn Vulnerable Web App}
Configure Security Level \texttt{medium} and try to exploit the hardened 
CSRF vulnerability.
Your exploit may consist of HTML markup and JavaScript code.
\end{subtask}

\begin{subtask}
\paragraph{TWiki}
Write an exploit which creates a new wiki page. To achieve this, prepare 
a HTML page with the embedded exploit. 
\end{subtask}


\task{Advanced CSRF}{60}
These tasks will be done using the advanced-csrf-exploitation web application.

\begin{subtask}
\paragraph{Token Synchronizer Pattern}
The site ... contains a form which has been made safe against CSRF attacks
using the Token Synchronizer Pattern.
Since the server does not hold any state, the pattern has been implemented
in a way that state is only saved on the client using cookies. 
Try to exploit the application in a way that when a visitor visits .../home
he automatically votes.

%two pages: vote page, home page with messages for other visitors.
%
%in conjuction with xss. scenario: token is saved in cookie and we use xss
%to read it, create the form, fill it in.
\end{subtask}

\begin{subtask}
\paragraph{Token Synchronizer Pattern}
The implementation of the Token Synchronizer Pattern has been improved.
Find a way to still exploit the web page.

\begin{comment}
``Usually, the tokens should be some function (with a secret key - known only
to the server; e.g., MAC) of the cookie! not the cookie.

Than the flow is as follows: 1. Client sends the server request with a
cookie. 2. Server returns a web page with CSRF token(s) for different
purposes (e.g., forms or just a simple get requests via the URL). 3. The
client performs some action (via POST or GET) and sends request with the
token (in the request body or in the URL) and with the cookie. 4. The
server is stateless, but it can verify that the request was sent by the
same client by calculating the function (with the secret key that the
server knows) on the cookie (or on part of it), and comparing the output
with the token.

In the case of CSRF, the cookie is automatically appended to the request by
the browser, but the attacker (that probably even doesn't know the cookie)
cannot add the corresponding tokens.''
\end{comment}

% a fixed token is generated.
% xss on other page
% create iframe to vote page and read the token
\end{subtask}

%\begin{subtask}
%referrer check
%The site from the previous task has been improved and now also checks the 
%\end{subtask}

\begin{subtask}
%multi step form with json of all answers on last step
\paragraph{Multi-Step Form}
The site ... contains a multi-step form of a survey. Your goal is to
prepare an exploit which will lead the browser who interprets the code
into submitting the completed survey.
\end{subtask}

\begin{subtask}
\paragraph{Semi-Blind, Multi-Step Form}
The site from the previous task has been improved in manners of security. 
However, it is still exploitable.
Develop an exploit which subsequently fills each step of the survey with 
data and eventually submits the completed survey.
%multi-step form with state on each step
%semi-blind
\end{subtask}

\end{document}
