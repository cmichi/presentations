% !TEX encoding = UTF-8 Unicode
\documentclass{report}

\title{Practical IT-Security: CSRF}
\author{Michael Müller}
\date{\today}
	
% big font for sections
\usepackage{sectsty}
\sectionfont{\LARGE}

\usepackage{graphicx}
\usepackage{wrapfig}
\usepackage{caption}
\usepackage{subcaption}
\usepackage{listings}

% \begin{comment} ... \end{comment{}
\usepackage{verbatim}

\setlength{\parskip}{0pt}

\makeatletter
\renewcommand{\paragraph}{
  \@startsection{paragraph}{4}
    {\z@}{1.25ex \@plus 1ex \@minus .2ex}{-1em}
      {\normalfont\normalsize\bfseries}
      }
      \makeatother


\begin{document}

\newpage

\maketitle

\newpage


%=================
\section{Introduction}

Cross-Site-Request-Forgery (\textsc{CSRF}) is an attack on web-applications
which will be further explained within the presentation.
This document is meant as a preparation document, to get you ready to
understand the presentation and conduct the assignments.

\subsection{HTTP}
The HTTP protocol is the driving force behind the web.

\subsubsection{GET}
foo ar



\begin{lstlisting}[
	caption=Typical HTTP GET request
]
GET / HTTP/1.1
Host: google.com
\end{lstlisting}

\begin{lstlisting}[
	caption=Typical HTTP GET reply
]
200 OK 
\end{lstlisting}

\subsubsection{POST}


%=================
\section{Assignments}

To carry out the assignments you need to bring your own laptop. 

\subsection{Damn Vulnerable Web App}
\subsubsection{Installation}
For the assignments we will use the Damn Vulnerable Web App (\textsc{DVWA}).
Download the package from \mbox{\url{http://www.dvwa.co.uk/}} and install it 
on your local machine. You will need a WebServer with PHP support and MySQL.

I advise against installing \textsc{DVWA} on a server/hosting provider with 
internet access. This may compromise other services on the same machine.
The better way is to set up a local webserver. The easiest way to do so
is probably to use XAMPP (Mac, Windows) or LAMP (Linux). Those packages
come with a webserver, pre-configured PHP support and MySQL.

\subsubsection{Test your installation}

\end{document}
