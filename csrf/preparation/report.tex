% !TEX encoding = UTF-8 Unicode
\documentclass{report}

\title{Practical IT-Security: CSRF}
\author{Michael Müller}
\date{\today}
	
% big font for sections
\usepackage{sectsty}
\sectionfont{\LARGE}

\usepackage{graphicx}
\usepackage{wrapfig}
\usepackage{caption}
\usepackage{subcaption}
\usepackage{listings}
\usepackage{hyperref}

% \begin{comment} ... \end{comment{}
\usepackage{verbatim}

\setlength{\parskip}{0pt}

\makeatletter
\renewcommand{\paragraph}{
  \@startsection{paragraph}{4}
    {\z@}{1.25ex \@plus 1ex \@minus .2ex}{-1em}
      {\normalfont\normalsize\bfseries}
      }
      \makeatother


\begin{document}

\newpage

\maketitle

\newpage


%=================
\section{Introduction}

Cross-Site-Request-Forgery (\textsc{CSRF}) is an attack on web-applications
which will be further explained within the presentation.
This document is meant as a preparation document in order to get you ready to
understand the presentation and conduct the assignments.

\begin{comment}
\subsection{HTTP}
The HTTP protocol is the driving force behind the web.

\subsubsection{GET}
foo ar



\begin{lstlisting}[
	caption=Typical HTTP GET request
]
GET / HTTP/1.1
Host: google.com
\end{lstlisting}

\begin{lstlisting}[
	caption=Typical HTTP GET reply
]
200 OK 
\end{lstlisting}

\subsubsection{POST}
\end{comment}


%=================
\subsection{Web Developer Tool}

We aim to exploit several services in the assignments. Most of the exploits
will be written in HTML in conjuction with JavaScript. For this purpose it
comes handy if you are familiar with the developer tool of some browser.
Most modern browsers come with some kind of developer tools. Check the
corresponding manual or install a fitting plugin like FireBug (for Firefox). 
For Chromium the developer tools are accessible via ``\emph{Tools > Developer 
Tools}'' (or via pressing \texttt{F12}).
The main features of the developer tools, which will come handy for us, are: 

\paragraph{The JavaScript Console}
Enables us to view JavaScript errors in exploits.

\paragraph{The Network Capturer}
Enables us to analyze and debug the HTTP requests and replies which area
executed by the browser.


%=================
\section{Assignments}

To carry out the assignments you need to bring your own laptop configured
to the run the services described in this section.

\subsection{Damn Vulnerable Web App}
\subsubsection{Installation}
For the assignments we will use the Damn Vulnerable Web App (\textsc{DVWA}).
Download the package from \href{http://www.dvwa.co.uk/}{the website} and 
install it on your local machine. You will need a WebServer with PHP support 
and MySQL. To install the package extract the project into a folder which
is configured to contain PHP code and point your browser to 
\href{http://localhost/dvwa/index.php}{http://localhost/dvwa/index.php}.

The documentation advises against installing \textsc{DVWA} on a server/hosting 
provider with internet access. This may compromise other services on the same 
machine.
The better way is to set up a local webserver. The easiest way to do so
is probably to use XAMPP (Mac, Windows) or LAMP (Linux). Those packages
come with a webserver, pre-configured PHP support and MySQL.

\subsubsection{Configuration}
After you have completed the installation you should log into the
application to get a feeling for the interface.

Open up \url{http://localhost/dvwa} and login with the credentials
User: \texttt{admin}, Password: \textt{password}.
Open the navigation tab
\texttt{DVWA Security} and configure \texttt{Security Level: Low}.

During the assignment we will work our way through the security levels 
\texttt{low} and \texttt{medium}. The level \texttt{high} is meant to show 
the state of the art best-practice solution. It is not meant to be cracked.\\

\textbf{Note:} The security level, which you configure is saved as a cookie 
value. The default value is \texttt{high}. This might lead to problems
while debugging: if you e.g. log into DVWA using Chromium and configure
\texttt{low} you should also run your exploit in Chromium and not 
in e.g. Firefox. Firefox would still be configured for the security level
\texttt{high}.


\section{Quick Checklist}

\paragraph{You should bring a basic understanding of}
\ \\
HTTP, HTML, the DOM, JavaScript.

\paragraph{You should bring}
\ \\
A laptop. Configured to run the Metasploitable image. Damn 
Vulnerable Web App should also be installed.

\paragraph{You should have played with}
\ \\
A Web Developer Tool.


\end{document}
