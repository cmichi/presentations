% !TEX encoding = UTF-8 Unicode
\documentclass{./handout}

\title{Javascript}
\assignmentno{6}

\date{Tuesday, 31 Jan. 2012, 23:59} % Abgabedatum

\begin{document}
\maketitle

\task{Javascript Basics}{8}

%\begin{subtask}
%What is Ajax? What is the difference between synchronous and asynchronous requests?
%\end{subtask}

\begin{subtask}
A desperate student asks for your help. Fix the code and make sure that it produces the output ''[Hello, World]".
\begin{lstlisting}
function Object(){
	this.elements = new Array("hello","bad","world");
	this.delete(index){
		delete this.elements[index];
	}
}
var o = new Object();
o.delete(1);
alert(o.elements);
\end{lstlisting}
\end{subtask}

%\begin{subtask}
%How is inheritance realized in Javascript? Provide an example.
%\end{subtask}

\begin{subtask}
Call the following script in such away that "800 600" is displayed in the dialog without changing the function.
\begin{lstlisting}
function byName(args){
	alert(args.x+" "+args.y);
}
\end{lstlisting}
\end{subtask}

\begin{subtask}
Javascript supports closures. Shortly explain the concept of closures. How are closures managed in memory?
\end{subtask}

\begin{subtask}
What is the output of the follow script? Why?
\begin{lstlisting}
alert("1" == 1);
alert("1" === 1);
alert(1 === 1);
alert("1" !== 1);
\end{lstlisting}
\end{subtask}

\begin{subtask}
Name at least two alternative ways of calling the method \texttt{o.m} without resetting the counter \texttt{cnt}.
% o.m.call(o);
% o.m.apply(o);
% o["m"]();
\begin{lstlisting}
function Object(){
	this.cnt=0;
	this.m=function(){
		alert("hello world: "+this.cnt);
		this.cnt++;
	}
} 
var o = new Object();
o.m();
\end{lstlisting}
\end{subtask}

\begin{subtask}
What is the difference between declaring variables with \texttt{var} and \texttt{let}?
\end{subtask}

\begin{subtask}
In Javascript the \texttt{this} statement  behaves different from other programming languages such as Java. Explain the difference.

The function \texttt{setTimeout(c,t)} calls a callback method \texttt{c} after $t/1000$s. Change the callback in the last line to output "hello" (from top.greeting!) after 1s.
%setTimeout(function(){
%	top.greet();
%},100);
\begin{lstlisting}
function ThisObjectProblem(){
	this.greeting = "hello";
	this.greet = function(){
		alert(this.greeting);
	}
}
var top = new ThisObjectProblem();
setTimeout(top.greet,1000);
\end{lstlisting}
\end{subtask}

\begin{subtask}
Convert the following object to a JSON string. Leave out attributes which cannot be represented in JSON.
% {"a":"hello","b":[1,2,3],"d":{"prop1":"web","prop2":"engineering"}}
 \begin{lstlisting}
function JSONObj(){
	this.a="hello";
	this.b =[1,2,3];
	this.c = function(){
		alert(2);
	}
	this.d = new Object();
	this.d.prop1 = "web";
	this.d["prop2"] = "engineering";
}

\end{lstlisting}
\end{subtask}

\pagebreak
\task{Event System}{2+4+6}

Javascript supports callback methods. They are frequently used in libraries, e.g. jQuery. One use case for callbacks is registering a callback method for an event and then react to it. The goal of this task is the implementation of such an event system in Javascript.
To make this task a little bit more \emph{educative}, you have to implement it object oriented with inheritance.

\begin{subtask}
First, familiarize yourself with inheritance in Javascript by implementing a class \texttt{ArrayWrapper} that contains an empty array and one method that returns the size of the array.
\end{subtask}

\begin{subtask}
Now, implement a class \texttt{EventSystem} that inherits from \texttt{ArrayWrapper}. The new class should provide the following functionality:
\begin{itemize}
	\item Provide a Singleton instance of \texttt{EventSystem}.
	Also provide a static variable \texttt{EventSystem.instance} to access the Singleton instance-object.
	\item Register events with \texttt{EventSystem.instance.on(EventString,callback)}. It should be possible to register multiple callback methods for the same event.
	\item Deregister events with \texttt{EventSystem.instance.remove(EventString)}.
	Here it will suffice when all registered events  on the particular String will be removed.
	\item Firing events with \texttt{EventSystem.instance.emit(EventString)}. An exception should be thrown if the specified event does not exist.
\end{itemize}
\end{subtask}

%
%//TODO Implement:
%
%
%function ArrayWrapper(){
%	this.array=[]
%	
%	this.getSize=function(){
%		return this.array.length;
%	}
%}
%
%function EventSystem(){
%	
%	this.on = function(event,function_){
%		if (this.array[event] == undefined){
%			this.array[event] = [];
%		}
%		this.array[event].push(function_);	
%	
%	}
%	
%	this.emit = function(event,args){
%		if (this.array[event]==undefined){
%			throw "No callback found";
%		}
%		for (callback in this.array[event]){
%			this.array[event][callback].apply(this.array[event][callback],args);
%		}
%		
%	}
%	
%	this.remove = function(event){
%		var index = this.array.indexOf(event);
%		this.array=this.array.splice(index,1);
%	}
%}
%
%
%EventSystem.prototype = new ArrayWrapper();
%EventSystem.prototype.constructor = EventSystem;
%
%EventSystem.instance = new EventSystem();


\begin{subtask}
Implement a horse racing game to test your event system. You can use the provided HTML template or create your own consisting of a start/reset button and two "horses", e.g., represented by images. In addition, two status fields show what the horses are doing (ready, running, finished). The user can bet on a specific horse by toggling a radio button. On pressing start the horses race across the screen and the user is informed if he has won or lost.

Implement the following features:
\begin{itemize}
	\item When pressing the start button, the function \texttt{animations\_start} is called and the button text is changed to "reset".
	\item When pressing the reset button, the function  \texttt{animations\_reset} is called and the button text is changed to "start".
	\item \texttt{animations\_start} lets the horses run, i.e., they are moved over the screen from left to right. The finish line are the coordinates \texttt{left:500px}. A horse should need 3-6s to cover the distance. There should never be a tie between the two horses (just to keep it simple).
	\item Once both horses reached the finish line, the user is informed if his horse won or lost.
	\item \texttt{animations\_reset} returns the horses to the start of the race track.
	\item The status fields should register for events at the horses, they should not be coupled with the animations functions.
	\item Each horse submits an event when it starts running, when it reached the finish line and when it is ready to run again (after reset).
\end{itemize}
You can use the jQuery (1.7) library for your game (\url{http://jquery.com}).
\end{subtask}
%var winner = null;
%var finished=0;
%$(document).ready(function(){
%	
%	//text
%	EventSystem.instance.on("anim1",function(){
%		$("#status_1").text("done");
%	});
%	EventSystem.instance.on("anim2",function(){
%		$("#status_2").text("done");
%	});
%	EventSystem.instance.on("reset",function(){
%		$("#status_1").text("stopped");
%		$("#status_2").text("stopped");
%	});
%	EventSystem.instance.on("start",function(){
%		$("#status_1").text("running");
%		$("#status_2").text("running");
%	});
%	
%	//ctrl
%	$("#ctrl").click(function(){
%		if ($(this).text()=="start"){
%			$(this).text("reset");
%			animations_start();
%		}else{
%			$(this).text("start");
%			animations_reset();
%		}
%		
%	});
%	
%	//winner
%	EventSystem.instance.on("anim1",function(){
%		if (winner == null){
%			winner = 1;
%		}
%		finished+=1;
%		if (finished>1){
%			finish();
%		}
%	});
%	EventSystem.instance.on("anim2",function(){
%		if (winner == null){
%			winner = 2;
%		}
%		finished+=1;
%		if (finished>1){
%			finish();
%		}
%	});
%});
%
%function getRandom(start,stop){
%	return Math.floor(Math.random()*((stop-start)*1000))+start*1000
%}
%function animations_start(){
%	var h1 = getRandom(2,6);
%
%	var h2 = getRandom(2,6);
%	while (h2==h1){
%		h2 += 500;
%	}
%	EventSystem.instance.emit("start");
%	$("#animation1").animate({left:500},h1,function(){
%		EventSystem.instance.emit("anim1");
%	});
%
%	$("#animation2").animate({left:500},h2,function(){
%		EventSystem.instance.emit("anim2");
%	});
%}
%function animations_reset(){
%	$("#animation1").css({left:30});
%	$("#animation2").css({left:30});
%	winner=null;
%	finished=0;
%	EventSystem.instance.emit("reset");
%	
%}
%function finish(){
%	var s1="you lose";
%	var s2="you win";
%	var a1;
%	var a2;
%	if (winner==1){
%		a1=s1;
%		a2=s2;
%	}else{
%		a1=s2;
%		a2=s1;
%	}
%	if ($("#rd1:checked").length==0){
%		alert(a1);
%	}else{
%		alert(a2);	
%	}	
%		
%}


\end{document}
